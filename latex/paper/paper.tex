
\documentclass{article}
\usepackage{hyperref}
\usepackage{enumitem}
\usepackage{graphicx}
\usepackage{amsmath}
\usepackage{mathpazo}
\usepackage{multicol}
\usepackage[a4paper,
            bindingoffset=0.2in,
            left=1in,
            right=1in,
            top=1in,
            bottom=1in,
            footskip=.25in]{geometry}

\begin{document}
\title{MS Technical Paper: \\ Placement Algorithms for Heterogenous FPGAs}
\author{Brian B Cheng \\ Rutgers University Department of Electrical and Computer Engineering}


\date{}
\maketitle

\section{Keywords}
\begin{itemize}
    \item FPGA, EDA, Placement, Simulated Annealing, Optimization, RapidWright
\end{itemize}


\begin{multicols}{2}
\section{Abstract}
    fdsafdsafdsa.

\section{Design Flow}
A typical FPGA design flow goes like the following. First, the developer writes the design entry in a high level Hardware Description Language (HDL) like Verilog or VHDL.
\begin{itemize}
\item Synthesis (Vivado default)
\item Placement (the focus of this paper)
    \begin{itemize}
        \item \textbf{Prepacking}:\\
        Traverse the raw EDIFNetlist from synthesis and identify patterns of cells that should be grouped together.
        This can be CARRY chains, DSP cascades, LUTFF pairs, etc.
    \item \textbf{Packing}:\\
        Take the prepacked cell groups and pack them into SiteInsts.
        Intra\-Site routing.
    \item \textbf{Placement}:\\
        Take the SiteInsts and optimally place them onto physical Sites.
    \end{itemize}
\item Routing
    Inter\-Site routing
\end{itemize}

\end{multicols}
\end{document}
