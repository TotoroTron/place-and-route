
\section{Packing}
\label{sec:packing}
Now that we have our \texttt{PrepackedDesign} object keeping track of multi-cell structures on the \texttt{edif} level, we can start packing them into \texttt{SiteInst} objects on the \texttt{design} level. 
Below are some of the most relevant classes from the \texttt{design} package for this task.
\begin{itemize}
    \item \texttt{Cell}: A cell corresponds to the leaf cell within the logical netlist \texttt{EDIFCellInst} and provides a mapping to a physical location BEL on the device. A cell can be created directly out of an \texttt{EDIFCellInst} to inherit all of its \texttt{edif} properties on the \texttt{design} level.
    \item \texttt{Net}: Represents the physical net to be routed (both inter-site and intra-site). When an \texttt{Cell} is created out of an \texttt{EDIFCellInst}, the \texttt{Net}s are automatically created out of its corresponding \texttt{EDIFNet}s.
    \item \texttt{SiteInst}: An instance of a \texttt{Site} on the \texttt{Device}. Carries the mapping information between the \texttt{BEL}s in a \texttt{Site} and the \texttt{Cell}s assigned to them. Also keeps track of the intra-Site routing information within (\texttt{Net}s, \texttt{SitePinInst}s, \texttt{SitePIP}s, etc.).
    \item \texttt{SitePIP}: A Programmable Interconnect Point (PIP) in a \texttt{Site}. Represents the fuses in intra-Site routing \texttt{BEL}s. 
    \item \texttt{SitePinInst}: An instance of a \texttt{SitePin} on a \texttt{Site}. These objects serve as the interface between intra-Site routing and general inter-Site routing. 
\end{itemize}

A \texttt{SiteInst} object can be created and its \texttt{BEL}s populated with existing \texttt{EDIFHierCellInst} objects from the \texttt{EDIFNetlist}. 
A \texttt{SiteInst} is placed on a specific \texttt{device} \texttt{Site} upon its creation using the constructor in Listing \ref{lst:siteinst}. 
The packer therefore assigns an initial placement to every \texttt{SiteInst} before the simulated‐annealing stage randomizes their positions. 
During this packing phase, we simply map \texttt{design} \texttt{SiteInst}s onto \texttt{device} \texttt{Site}s in coordinate order as they are generated (e.g., the first \texttt{SiteInst} onto \texttt{SLICEL\_X0Y0}, the second onto \texttt{SLICEL\_X0Y1}, and so on). 

% % One can think of a \texttt{design} \texttt{SiteInst} as a movable square peg and the corresponding \texttt{device} \texttt{Site} as a fixed square hole. 
% % Both \texttt{SiteInst}s and \texttt{Site}s come in various shapes or types (\texttt{SLICEL}, \texttt{SLICEM}, \texttt{DSP48E1}, etc.) and placement is only allowed between compatible pairs. 
% % For example, a \texttt{RAMB48E1} \texttt{SiteInst} can only be placed on a \texttt{RAMB48E1} \texttt{Site}.
% % A \texttt{DSP48E1} \texttt{SiteInst} can only be placed on a \texttt{DSP48E1} \texttt{Site}.
% % A \texttt{SLICEL} \texttt{SiteInst} may occupy either a \texttt{SLICEL} or a \texttt{SLICEM} \texttt{Site}, whereas a \texttt{SLICEM} \texttt{SiteInst} can only be placed on a \texttt{SLICEM} \texttt{Site}.
% % These constraints must always be followed as the \texttt{SiteInst}s move across the device throughout the placement stage to ensure legality. 

Listing \ref{lst:siteinst} shows how to use the \texttt{SiteInst} object and how it ties \texttt{Site}s, \texttt{Cell}s, \texttt{BEL}s, and intra-Site \texttt{Net}s together. 
The super-specifics of the packing process via RapidWright is too involved for the scope of this paper, but just to give an impression of how the \texttt{design} classes are used, Listing \ref{lst:single_carry_chains}, \ref{lst:carry_chains}, \ref{lst:carry_nets}, and \ref{lst:ff_nets} together show how \texttt{CarryCellGroup}s are packed into \texttt{SLICEL} \texttt{Sites} in our working code. 

Packing the other \texttt{Cell} structures into \texttt{Site}s follow a similar process. 
For each cell cluster, we create a \texttt{SiteInst} and populate its \texttt{BEL}s with the corresponding \texttt{Cell}s in the cluster, assign it a physical \texttt{Site} on the \texttt{device}.
We further consolidate these \texttt{SiteInst}s into clusters or lists of \texttt{SiteInst}s for multi-Site structures like \texttt{CarryCellGroup} chains and DSP cascades. 
For each \texttt{SiteInst}, we assign it a physical \texttt{Site} on the \texttt{device} and route its internal intra-Site nets. 
The end result is a \texttt{PackedDesign} object which is a group of data structures solely around \texttt{SiteInst}s like shown in figure \ref{fig:substages}.
At this point we no longer need to worry about individual \texttt{BEL}s, \texttt{Cell}s, or intra-Site \texttt{Nets} as they have now been packaged and taped-off into their respective \texttt{SiteInst}s. 
These \texttt{SiteInst} become the new atomic level components in the following Placement stage. 

\end{multicols}
\begin{lstlisting}[caption={\texttt{SiteInst} constructor and methods.}, label={lst:siteinst}]
SiteInst Constructor: 
    SiteInst(String name, Design design, SiteTypeEnum type, Site site)

Most relevant SiteInst Methods:
    createCell(EDIFHierCellInst inst, BEL bel) // Populating the SiteInst BELs with Cells
    unplace() // Unplacing the SiteInst from its current Site
    place(Site site) // Placing an unplaced SiteInst onto a Site
    routeSite() // Attempt to automatically route all intra-Site nets (manual intervention likely required)
    routeIntraSiteNet(Net net, BELPin src, BELPin snk) // Manually route an intra-Site net

Example:
    SiteInst si = new SiteInst("mySiteInst", design, SiteTypeEnum.SLICEL, device.getSite("SLICEL_X0Y1"));
    si.createCell(someFDRECell, si.getBEL("AFF"));
    si.routeSite();
    si.unplace();
    si.place(device.getSite("SLICEL_X15Y33"));
    // In Simulated Annealing, SiteInst objects will be unplaced() and placed() many times to converge to an optimal solution. 
    // All Cell-BEL mapping and intra-Site routing is preserved when a SiteInst is moved. 

\end{lstlisting}


\begin{lstlisting}[language=java, caption={Packing one \texttt{CarryCellGroup} into one \texttt{SLICEL} \texttt{SiteInst}.}, label={lst:single_carry_chains}]
// Names of the BELs in the 4 LUT-FF lanes:
protected String[] FF_BELS = new String[] { "AFF", "BFF", "CFF", "DFF" };
protected String[] LUT6_BELS = new String[] { "A6LUT", "B6LUT", "C6LUT", "D6LUT" };
// Pack one CarryCellGroup 
private void packCarrySite(CarryCellGroup carryCellGroup, SiteInst si) {
    // Iterate through 4 LUT-FF Lanes
    for (int i = 0; i < 4; i++) {
        EDIFHierCellInst ff = carryCellGroup.ffs().get(i);
        if (ff != null)
            si.createCell(ff, si.getBEL(FF_BELS[i]));
        EDIFHierCellInst lut = carryCellGroup.luts().get(i);
        if (lut != null)
            si.createCell(lut, si.getBEL(LUT6_BELS[i]));
        // carry site LUTs MUST be placed on LUT6 BELs.
        // only LUT6/O6 can connect to CARRY4/S0
    }
    si.createCell(carryCellGroup.carry(), si.getBEL("CARRY4"));
    // default intrasite routing
    si.routeSite();
    // sometimes the default routeSite() is insufficient, so some manual intervention is required
    rerouteCarryNets(si);
    rerouteFFClkSrCeNets(si);
} // end placeCarrySite()
\end{lstlisting}

\newpage
\begin{lstlisting}[language=java, caption={Packing \texttt{CarryCellGroup}s into \texttt{SLICEL} \texttt{SiteInst}s.}, label={lst:carry_chains}]
// Pack all CarryCellGroup chains
private List<List<SiteInst>> packCarryChains(List<List<CarryCellGroup>> EDIFCarryChains)
        throws IOException {
    List<List<SiteInst>> siteInstChains = new ArrayList<>();
    writer.write("\n\nPacking carry chains... (" + EDIFCarryChains.size() + ")");
    for (List<CarryCellGroup> edifChain : EDIFCarryChains) {
        List<SiteInst> siteInstChain = new ArrayList<>();
        writer.write("\n\t\tChain Size: (" + edifChain.size() + "), Chain Anchor: "
                + edifChain.get(0).carry().getFullHierarchicalInstName());
        Site anchorSite = selectCarryAnchorSite(edifChain.size());
        SiteTypeEnum selectedSiteType = anchorSite.getSiteTypeEnum();
        for (int i = 0; i < edifChain.size(); i++) {
            Site site = (i == 0) ? anchorSite
                    : device.getSite("SLICE_X" + anchorSite.getInstanceX() + "Y" + (anchorSite.getInstanceY() + i));
            SiteInst si = new SiteInst(edifChain.get(i).carry().getFullHierarchicalInstName(), design,
                    selectedSiteType,
                    site);
            packCarrySite(edifChain.get(i), si);
            if (i == 0) { // additional routing logic for anchor site
                Net CINNet = si.getNetFromSiteWire("CIN");
                CINNet.removePin(si.getSitePinInst("CIN"));
                si.addSitePIP(si.getSitePIP("PRECYINIT", "0"));
            }
            occupiedSites.get(selectedSiteType).add(site);
            availableSites.get(selectedSiteType).remove(site);
            siteInstChain.add(si);
        }
        siteInstChains.add(siteInstChain);
    } // end for (List<EDIFCellInst> chain : EDIFCarryChains)
    return siteInstChains;
} // end packCarryChains()
\end{lstlisting}


\begin{lstlisting}[language=java, caption={Manually rerouting intra-Site nets in a \texttt{SLICEL} containing a \texttt{CARRY4} \texttt{Cell}}, label={lst:carry_nets}]
protected String[] FF_BELS = new String[] { "AFF", "BFF", "CFF", "DFF" };
private void rerouteCarryNets(SiteInst si) {
    // activate PIPs for CARRY4/COUT
    si.addSitePIP(si.getSitePIP("COUTUSED", "0"));
    // undo default CARRY4/DI nets
    SitePinInst AX = si.getSitePinInst("AX");
    if (AX != null)
        si.unrouteIntraSiteNet(AX.getBELPin(), si.getBELPin("ACY0", "AX"));
    SitePinInst DX = si.getSitePinInst("DX");
    if (DX != null)
        si.unrouteIntraSiteNet(DX.getBELPin(), si.getBELPin("DCY0", "DX"));
    // activate PIPs for CARRY4/DI pins
    si.addSitePIP(si.getSitePIP("DCY0", "DX"));
    si.addSitePIP(si.getSitePIP("CCY0", "CX"));
    si.addSitePIP(si.getSitePIP("BCY0", "BX"));
    si.addSitePIP(si.getSitePIP("ACY0", "AX"));
    // remove stray CARRY4/CO nets
    if (si.getNetFromSiteWire("CARRY4_CO2") != null)
        design.removeNet(si.getNetFromSiteWire("CARRY4_CO2"));
    if (si.getNetFromSiteWire("CARRY4_CO1") != null)
        design.removeNet(si.getNetFromSiteWire("CARRY4_CO1"));
    if (si.getNetFromSiteWire("CARRY4_CO0") != null)
        design.removeNet(si.getNetFromSiteWire("CARRY4_CO0"));
    // add default XOR PIPs for unused FFs
    for (String FF : FF_BELS)
        if (si.getCell(FF) == null)
            si.addSitePIP(si.getSitePIP(FF.charAt(0) + "OUTMUX", "XOR"));
} // end rerouteCarryNets()
\end{lstlisting}

\begin{lstlisting}[language=java, caption={Manually reroute intra-Site nets in a \texttt{SLICEL} containing FF \texttt{Cells}}, label={lst:ff_nets}]
private void rerouteFFClkSrCeNets(SiteInst si) {
    si.addSitePIP("CLKINV", "CLK");
    // activate PIPs for SR and CE pins
    Net SRNet = si.getNetFromSiteWire("SRUSEDMUX_OUT");
    Net CENet = si.getNetFromSiteWire("CEUSEDMUX_OUT");
    // if routeSite() default PIPs are incorrect, deactivate them then activate correct PIP
    if (SRNet != null) {
        if (SRNet.isGNDNet()) {
            if (si.getSitePIPStatus(si.getSitePIP("SRUSEDMUX", "IN")) == SitePIPStatus.ON)
                for (String FF : FF_BELS)
                    si.unrouteIntraSiteNet(si.getSitePinInst("SR").getBELPin(), si.getBELPin(FF, "SR"));
            si.addSitePIP(si.getSitePIP("SRUSEDMUX", "0"));
        } else {
            if (si.getSitePIPStatus(si.getSitePIP("SRUSEDMUX", "0")) == SitePIPStatus.ON)
                for (String FF : FF_BELS)
                    si.unrouteIntraSiteNet(si.getBELPin("SRUSEDGND", "0"), si.getBELPin(FF, "SR"));
            si.addSitePIP(si.getSitePIP("SRUSEDMUX", "IN"));
        }
    }
    if (CENet != null) {
        if (CENet.isVCCNet()) {
            if (si.getSitePIPStatus(si.getSitePIP("CEUSEDMUX", "IN")) == SitePIPStatus.ON)
                for (String FF : FF_BELS)
                    si.unrouteIntraSiteNet(si.getSitePinInst("CE").getBELPin(), si.getBELPin(FF, "CE"));
            si.addSitePIP(si.getSitePIP("CEUSEDMUX", "1"));
        } else {
            if (si.getSitePIPStatus(si.getSitePIP("CEUSEDMUX", "1")) == SitePIPStatus.ON)
                for (String FF : FF_BELS)
                    si.unrouteIntraSiteNet(si.getBELPin("CEUSEDGND", "1"), si.getBELPin(FF, "CE"));
            si.addSitePIP(si.getSitePIP("CEUSEDMUX", "IN"));
        }
    }
} // end rerouteFFSrCeNets()
\end{lstlisting}
\begin{multicols}{2}





