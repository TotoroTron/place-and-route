
\section{What is a Netlist?}
\label{sec:netlist}
In its most general form, a netlist is a list of every component in an electronic design paired with a list of nets they connect to. 
Depending on the working abstraction level, these components can be transistors, logic gates, macrocells, or increasingly higher-level modules. 
Generally, a net denotes any group of two or more interconnected components.
In an electronics context, a net can be though of as a wire connecting multiple pins between multiple components, with each wire having one voltage source and one or more voltage sinks. 
Thus, if one were to express the netlist as a graph, the graph would consist of nodes (components) connected by hyperedges (wires). 
More precisely, these hyperedges connect the pins between the components, not the components themselves, with each component exposing multiple pins or ports. 

In FPGA context, the components are logical cells (\texttt{LUTs}, \texttt{CARRY4s}, etc.) or hierarchical cells (Verilog module instances) with pins connected together by wires. 
In Vivado, a Netlist can be synthesized as a Hierarchical or a Flattened netlist. 
Figure ~\ref{fig:hierarchical_design} shows an example a Verilog design using module instance hierarchy, while figure ~\ref{fig:hier_netlist} shows the same design synthesized into a flattened netlist. 
The synthesizer attempts to construct the module hierarchy as close to the module instantiation hierarchy as defined by the raw user design entry. 
Figure ~\ref{fig:flat_netlist} shows the same design but synthesized into a flat netlist. 
In either synthesized Netlist, the leaf cells, (deepest level cells), must necessarily consist only of primitive cells from the architecture primitives library (\texttt{LUT6}, \texttt{FDRE}, \texttt{CARRY4}, \texttt{DSP48E1}, etc.). 
The netlist can be compiled and exported as a purely structural low-level Verilog file, or an Electrinic Design Interchange Format (EDIF) file, both describing the netlist explicitly as a list of logical cells connected by a list of wires. 

\end{multicols}
{
    \raggedright
    \includegraphics[valign=t, scale=0.3]{figures/netlist_synth/top_level.png}
    \includegraphics[valign=t, scale=0.3]{figures/netlist_synth/module_0.png}
    \includegraphics[valign=t, scale=0.3]{figures/netlist_synth/module_1.png}
    \includegraphics[valign=t, scale=0.3]{figures/netlist_synth/module_2.png}
    \includegraphics[valign=t, scale=0.3]{figures/netlist_synth/module_3.png}
    \captionof{figure}{A simple HDL design with module hierarchy.}
    \label{fig:hierarchical_design}
}
{
    \centering
    \includegraphics[valign=c, width=11.5cm]{figures/netlist_synth/hier_netlist.png}
    \includegraphics[valign=c, width=6cm]{figures/netlist_synth/hier_graph.png}
    \captionof{figure}{
        \textbf{Left:} A hierarchical netlist consisting of LUTs and FFs.
        \textbf{Right:} The cell hierarchy graph.
    }
    \label{fig:hier_netlist}
}
\vspace{0.5cm}
{
    \centering
    \includegraphics[valign=c, width=10cm]{figures/netlist_synth/flat_netlist.png}
    \includegraphics[valign=c, width=4cm]{figures/netlist_synth/flat_graph.png}
    \captionof{figure}{
        \textbf{Left:} A flattened netlist consisting of LUTs and FFs.
        \textbf{Right:} The flattened cell hierarchy graph.
    }
    \label{fig:flat_netlist}
}
\begin{multicols}{2}

\subsection{Netlist Traversal and Manipulation in RapidWright}

One of RapidWright's most powerful features is netlist manipulation via the \texttt{edif} package of classes. A netlist can be easily extracted from a \texttt{.dcp} design checkpoint and traversed like the following: 

\begin{lstlisting}[language=Java, caption={Netlist extraction}, label={lst:netlist_extract}]
Design design = Design.readCheckpoint("synth.dcp")
EDIFNetlist = design.getNetlist();
\end{lstlisting}

The \texttt{Netlist} class in RapidWright stores hierarchical information about the \texttt{Design} and can be accessed and manipulated

SHOULD I INCLUDE A RAPIDWRIGHT HELLO WORLD EXAMPLE JUST TO GIVE AN IMPRESSION?

OR CAN I JUST DIRECT THE READER TO THE RAPIDWRIGHT DOCUMENTATION?

