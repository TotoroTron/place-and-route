
\section{RapidWright API}
\label{sec:rapidwright_api}

\textbf{RapidWright} is an open-source Java framework from AMD/Xilinx that provides direct access to the netlist and device databases used by vendor tools. 
This framework positions itself as an additional workflow column, allowing users to intercept or replace stages of the standard design flow with custom optimization stages (see Figure~\ref{fig:vivado_dcps}).

\begin{itemize}
\item \textbf{Design Checkpoints:} 
    RapidWright leverages \texttt{.dcp} files (design checkpoints) generated at various stages of a Vivado flow. 
    By importing a checkpoint, engineers can manipulate the netlist, placement, or routing externally, then re-export a modified checkpoint for further processing in the Vivado workflow column.

\item \textbf{Key Packages:} 
    RapidWright revolves around three primary data model packages:
    \begin{enumerate}
    \item \texttt{edif} -- Represents the logical netlist in an abstracted EDIF-like structure.
    \item \texttt{design} -- Contains data structures for the physical implementation (Cells, Nets, Sites, BELs, etc.).
    \item \texttt{device} -- Provides a database of the target FPGA architecture (e.g., Site coordinates, Tile definitions, routing resources).
    \end{enumerate}

\item \textbf{Interfacing with the Netlist and Device:} 
    An engineer can query the netlist to find specific resources (LUTs, FFs, DSPs, etc.) and then map or move them onto device sites. 
    This level of control over backend resources is necessary for research in custom placement, advanced packing techniques, or experimental routing algorithms.
\end{itemize}

{
    \centering
    \includegraphics[width=0.8\columnwidth]{figures/vivado_dcps.png}
    \captionof{figure}{RapidWright workflow integrating into the default Vivado design flow.}
    \label{fig:vivado_dcps}
}

By exposing these low-level internals, RapidWright allows fine-grained design transformations that go beyond the standard Vivado IDE’s capabilities. 
Researchers can prototype new EDA strategies without needing to re-implement an entire FPGA backend from scratch, thus accelerating innovation in placement and routing methodologies.

\subsection{What is a Netlist?}
\label{subsec:netlist}
In its most general form, a netlist is a list of every component in an electronic design paired with a list of nets they connect to. 
Depending on the abstraction level at hand, these components can be transistors, logic gates, macrocells, or increasingly higher-level modules. 
Generally, a net denotes any group of two or more interconnected components.
In an electronics context, a net can be though of as a wire connecting multiple pins between multiple components, with each wire having one voltage source and one or more voltage sinks. 
Thus, one could express the netlist as a hypergraph, nodes representing components, hyperedges representing wires connecting two or more component. 
More precisely, these hyperedges connect the ports on the components, not the components themselves, with each component exposing multiple ports. 

In FPGA context, the components are logical cells (\texttt{LUTs}, \texttt{CARRY4s}, etc.) or hierarchical cells (Verilog module instances) with pins connected together by wires. 
In Vivado, a Netlist can be synthesized as a Hierarchical or a Flattened netlist. 
Figure ~\ref{fig:hierarchical_design} shows an example a Verilog design with modules instantiated in a hierarchy. 
Figure ~\ref{fig:hier_netlist} shows the design synthesized into a \textbf{hierarchical netlist} with \textbf{hierarchical cells} and \textbf{leaf cells}. 
The synthesizer attempts to construct the module hierarchy as close to the module instantiation hierarchy defined by the user design entry. 
Figure ~\ref{fig:flat_netlist} shows the same design but synthesized into a \textbf{flattened netlist} containing \textbf{only leaf cells}. 

In either synthesized netlist, the \textbf{leaf cells}, (deepest level cells), must necessarily consist only of \textbf{primitive cells} from the architecture's primitive cell library (\texttt{LUT6}, \texttt{FDRE}, \texttt{CARRY4}, \texttt{DSP48E1}, etc.). 
The netlist can be compiled and exported as a purely structural low-level Verilog file, or an Electrinic Design Interchange Format (EDIF) file, both describing the netlist explicitly as a list of cell instances connected by a list of wires. 

\newpage
\end{multicols}
{
    \raggedright
    \includegraphics[valign=t, scale=0.3]{figures/netlist_synth/top_level.png}
    \includegraphics[valign=t, scale=0.3]{figures/netlist_synth/module_0.png}
    \includegraphics[valign=t, scale=0.3]{figures/netlist_synth/module_1.png}
    \includegraphics[valign=t, scale=0.3]{figures/netlist_synth/module_2.png}
    \includegraphics[valign=t, scale=0.3]{figures/netlist_synth/module_3.png}
    \captionof{figure}{A simple HDL design with module hierarchy.}
    \label{fig:hierarchical_design}
}
\vspace{0.5cm}
{
    \centering
    \includegraphics[valign=c, width=11.5cm]{figures/netlist_synth/hier_netlist.png}
    \includegraphics[valign=c, width=6cm]{figures/netlist_synth/hier_graph.png}
    \captionof{figure}{
        \textbf{Left:} A hierarchical netlist consisting of LUTs and FFs.
        \textbf{Right:} The cell hierarchy tree.
    }
    \label{fig:hier_netlist}
}
\vspace{0.5cm}
{
    \centering
    \includegraphics[valign=c, width=13.0cm]{figures/netlist_synth/flat_netlist.png}
    \includegraphics[valign=c, width=4cm]{figures/netlist_synth/flat_graph.png}
    \captionof{figure}{
        \textbf{Left:} A flattened netlist consisting of LUTs and FFs.
        \textbf{Right:} The flattened cell hierarchy tree.
    }
    \label{fig:flat_netlist}
}
\newpage
\begin{multicols}{2}

\subsection{Netlist Traversal and Manipulation via RapidWright}

RapidWright represents the logical netlist objects via the \texttt{edif} classes:
\begin{itemize}
    \item \texttt{EDIFNetlist}: The full logical netlist of a \texttt{Design}. 
    \item \texttt{EDIFNet}: A logical net within an \texttt{EDIFNetlist}.
    \item \texttt{EDIFHierNet}: Combines an \texttt{EDIFNet} with a full hierarchical instance name to uniquely identify a net in a netlist.
    \item \texttt{EDIFCell}: A logical cell in an \texttt{EDIFNetlist}. 
    \item \texttt{EDIFCellInst}: An instance of an \texttt{EDIFCell}. 
    \item \texttt{EDIFHierCellInst}: An \texttt{EDIFCellInst} with its hierarchy, described by all the \texttt{EDIFCellInst}s that sit above it within the netlist.
    \item \texttt{EDIFPort}: A port on an \texttt{EDIFCell}. 
    \item \texttt{EDIFPortInst}: An instance of a port on an \texttt{EDIFCellInst}. 
    \item \texttt{EDIFHierPortInst}: Combines an \texttt{EDIFPortInst} with a full hierarchical instance name to uniquely identify a port instance in a netlist. 
\end{itemize}


Using these classes and their associated methods, we can traverse the logical netlist (\texttt{EDIFNetlist}) and analyze or manipulate it as we see fit. 
A netlist can be easily extracted from a \texttt{.dcp} design checkpoint file and traversed like shown in Listing \ref{lst:netlist_extract}. 
This is performed on the same design shown in figure \ref{fig:flat_netlist}.

\begin{lstlisting}[language=java, caption={Basic netlist extraction and traversal}, label={lst:netlist_extract}]
Design design = Design.readCheckpoint("synth.dcp")
EDIFNetlist netlist = design.getNetlist();

// Example task:
// Extract the set of all unique nets from the design.

// Initialize a new Set:
Set<EDIFNet> netSet = new HashSet<>();

// Access all leaf cells
List<EDIFCellInst> ecis = netlist.getAllLeafCellInstances();

// Traverse the cell list
for (EDIFCellInst eci : ecis) {
    // Access the ports on this cell
    Collection<EDIFPortInst> epis = eci.getPortInsts();
    for (EDIFPortInst epi : epis) {
        // Access the net on this port
        EDIFNet net = epi.getNet();
        netSet.add(net);
    }
}

// Downstream task:
// For each unique net, print out the incident cells.

// Traverse the set of nets
for (EDIFNet net : netSet) {
    System.out.println("Net: " + net.getName());
    // Access the ports connected to this net
    Collection<EDIFPortInst> epis = net.getPortInsts();
    for (EDIFPortInst epi : epis) {
        // Access the cell that this port belongs to
        EDIFCellInst eci = epi.getCellInst();
        if (eci == null) {
            // (top_level ports have no associated cell)
            continue;
        } else {
            System.out.println(
                "\tCell: " + eci.getName() + 
                ",\tCellType: " + eci.getCellName()
            );
        }
    }
}
\end{lstlisting}

{
    \centering
    \includegraphics[valign=c, width=\columnwidth]{figures/traversal.png}
    \captionof{figure}{\texttt{Netlist} traversal via the \texttt{EDIFCellInst}, \texttt{EDIFPortInst}, and \texttt{EDIFNet} classes}
    \label{fig:traversal}
}
\vspace{0.5cm}

In this method, we get the list of all \texttt{EDIFCellInst} from the \texttt{EDIFNetlist} and traverse the list of cells and access their connected nets. 
Alternatively, we could have also extracted the list of all \texttt{EDIFNet}s directly from the \texttt{design} and traversed the nets to access their connected cells. 

\begin{lstlisting}[caption={Code Printout}]
Net: dout[0]
	Port: I0, Cell: dout_i_1, CellType: LUT4
	Port: Q, Cell: m0/dout_reg, CellType: FDRE
Net: q_1
	Port: I2, Cell: dout_i_1, CellType: LUT4
	Port: Q, Cell: m0/q_1_reg, CellType: FDRE
	Port: I5, Cell: q_1_i_1, CellType: LUT6
Net: ce
	Port: I1, Cell: dout_i_1, CellType: LUT4
	Port: I1, Cell: dout_i_1__0, CellType: LUT6
	Port: I1, Cell: dout_i_1__1, CellType: LUT6
	Port: I3, Cell: q_1_i_1, CellType: LUT6
Net: dout[1]
	Port: I0, Cell: dout_i_1__0, CellType: LUT6
	Port: Q, Cell: m1/m2/dout_reg, CellType: FDRE
Net: dout_i_1__1_n_0
	Port: O, Cell: dout_i_1__1, CellType: LUT6
	Port: D, Cell: m1/m3/dout_reg, CellType: FDRE
Net: clk
	Port: C, Cell: m0/dout_reg, CellType: FDRE
	Port: C, Cell: m0/q_1_reg, CellType: FDRE
	Port: C, Cell: m1/m2/dout_reg, CellType: FDRE
	Port: C, Cell: m1/m3/dout_reg, CellType: FDRE
Net: dout[2]
	Port: I0, Cell: dout_i_1__1, CellType: LUT6
	Port: Q, Cell: m1/m3/dout_reg, CellType: FDRE
Net: <const0>
	Port: G, Cell: GND, CellType: GND
	Port: R, Cell: m0/dout_reg, CellType: FDRE
	Port: R, Cell: m0/q_1_reg, CellType: FDRE
	Port: R, Cell: m1/m2/dout_reg, CellType: FDRE
	Port: R, Cell: m1/m3/dout_reg, CellType: FDRE
Net: <const1>
	Port: P, Cell: VCC, CellType: VCC
	Port: CE, Cell: m0/dout_reg, CellType: FDRE
	Port: CE, Cell: m0/q_1_reg, CellType: FDRE
	Port: CE, Cell: m1/m2/dout_reg, CellType: FDRE
	Port: CE, Cell: m1/m3/dout_reg, CellType: FDRE
Net: dout_i_1__0_n_0
	Port: O, Cell: dout_i_1__0, CellType: LUT6
	Port: D, Cell: m1/m2/dout_reg, CellType: FDRE
Net: rst
	Port: I3, Cell: dout_i_1, CellType: LUT4
	Port: I5, Cell: dout_i_1__0, CellType: LUT6
	Port: I5, Cell: dout_i_1__1, CellType: LUT6
	Port: I4, Cell: q_1_i_1, CellType: LUT6
Net: dinc
	Port: I2, Cell: dout_i_1__0, CellType: LUT6
	Port: I2, Cell: dout_i_1__1, CellType: LUT6
	Port: I2, Cell: q_1_i_1, CellType: LUT6
Net: dinb
	Port: I3, Cell: dout_i_1__0, CellType: LUT6
	Port: I4, Cell: dout_i_1__1, CellType: LUT6
	Port: I0, Cell: q_1_i_1, CellType: LUT6
Net: dina
	Port: I4, Cell: dout_i_1__0, CellType: LUT6
	Port: I3, Cell: dout_i_1__1, CellType: LUT6
	Port: I1, Cell: q_1_i_1, CellType: LUT6
Net: q_1_i_1_n_0
	Port: D, Cell: m0/q_1_reg, CellType: FDRE
	Port: O, Cell: q_1_i_1, CellType: LUT6
Net: dout_i_1_n_0
	Port: O, Cell: dout_i_1, CellType: LUT4
	Port: D, Cell: m0/dout_reg, CellType: FDRE
\end{lstlisting}


\subsection{Placement Flow via RapidWright}
\label{subsec:placement}
\end{multicols}
{
    \centering
    \includegraphics[width=0.8\columnwidth]{figures/substages.png}
    \captionof{figure}{The data classes populated at each substage: \texttt{PrepackedDesign}, \texttt{PackedDesign}, and \texttt{PlacedDesign}.}
    \label{fig:substages}
}
\begin{multicols}{2}
\label{sec:simulated_annealing}
With a basic understanding of FPGA architecture, design placement, and RapidWright, we have all the necessary pieces to implement our SA placer. 
Here we outline in detail each substage of our implementation: PrePacking, Packing, and Placement. 
Shown in Figure \ref{fig:edif_design_device} is an overview of the placement workflow. 
Figure \ref{fig:substages} shows the data structures of RapidWright objects that are populated at each stage: \texttt{PrepackedDesign}, which is a group of data structures around \texttt{EDIFHierCellInst}s, \texttt{PackedDesign}, which is a group fo data structures around \texttt{SiteInst}s, and finally, \texttt{PlacedDesign}, which is simply captured by the final RapidWright \texttt{Design} object. 

\vspace{0.5cm}
{
    \centering
    \includegraphics[width=0.9\columnwidth]{figures/edif_design_device.png}
    \captionof{figure}{Our placement workflow}
    \label{fig:edif_design_device}
}


