\section{Future Work}


\subsection{Improve Robustness}
In this paper we have made many simplifications to the problem space just to make the placer easier to implement.
For example, our placer in its current state does not take advantage of \texttt{SLICEL}/\texttt{SLICEM} heterogeneity and simply maps all SLICE \texttt{SiteInst}s onto \texttt{SLICEL}s. 
Recall that the SLICE Sites in Xilinx FPGAs typically come in a 75-25\% split between \texttt{SLICEL}s and \texttt{SLICEM}s. 
This means that we have rendered about 25\% of the CLB fabric unusable which will inevitably hurt wirelength minimization during placement since the \texttt{SiteInst}s must be spread over a larger area. 
Enabling \texttt{SLICEL}-\texttt{SLICEM} heterogeneity can lead do greater logic density and consequently less total HPWL, but can make the packing process more complex and may contribute to higher routing congestion.

We can also add packing support for other Xilinx primitives such as the shift-register primitive \texttt{SRLx}, distributed RAM primitives \texttt{RAMSx}, or even \texttt{LATCH} primitives as discussed in \ref{sec:7_series}.
Adding support for additional primitives and macros will allow our placer to handle a wider range of HDL designs and will require deeper consideration of hardware constraints to ensure robustness.

In its current state, the prepacker and packer struggle to handle signals larger than 24-bits, especially when involved in DSP functions like addition and multiplication. 
In such designs, the Vivado synthesizer may synthesize long \texttt{CARRY4} chains with particular \texttt{EDIFHierPortInst} configurations which are currently not handled by our packer and lead to failures in the subsequent routing stage.
Further work is required to resolve these constraints in the packer.

\subsection{Force-Directed and Analytical Placement}
As discussed earlier, simulated annealing (SA) has largely been phased out in state-of-the-art (SOTA) placers due to its poor scalability and long runtimes. 
Our current placer follows a straightforward \emph{prepack–pack–place} flow, with the final placement stage driven by SA. 
In future work, this stage can be retrofitted to use analytical placement (AP) while reusing the prepacker and packer. 
Here, \texttt{SiteInst}s remain the atomic placement objects, and analytical solvers can be applied to minimize the total half-perimeter wirelength (HPWL) of the connecting nets.

The following works provide a foundation for understanding AP, ranging from early formulations to modern SOTA methods:
\begin{itemize}
    \item \emph{Analytical minimization of half-perimeter wirelength} - \textbf{Kennings and Markov (2000)} \cite{AP_2000}.
    \item \emph{Kraftwerk2—A Fast Force-Directed Quadratic Placement Approach Using an Accurate Net Model} - \textbf{Spindler et al. (2008)} \cite{kraftwerk2}
    \item \emph{Analytical placement for heterogeneous FPGAs} - \textbf{Gort et al. (2012)} \cite{AP_2012}.
    \item \emph{SimPL: An algorithm for placing VLSI circuits} - \textbf{Kim et al. (2013)} \cite{SimPL}.
    \item \emph{Multi-Electrostatic FPGA Placement Considering SLICEL–SLICEM Heterogeneity, Clock Feasibility, and Timing Optimization} - \textbf{Jing et al. (2023)} \cite{MultiElectrostatic}.
    \item \emph{OpenPARF 3.0: Robust Multi-Electrostatics Based FPGA Macro Placement Considering Cascaded Macro Groups and Fence Regions} - \textbf{Jing et al. (2024)} \cite{OpenPARF}.
\end{itemize}

As noted in Section~\ref{sec:placement}, replacing SA with AP requires splitting the placement stage into two substages:  
\emph{global placement}, which determines continuous target positions using analytical optimization, and  
\emph{detailed placement} (legalization), which snaps objects to valid \texttt{site}s while adhering to hardware constraints.  
Global placement can be performed using any number of off-the-shelf (OTS) analytical solvers.

Unlike SA, which primarily relies on heuristics and probabilistic moves, AP begins by contextualizing the problem as a collection of math expressions that are compatible with OTS solvers.
As briefly introduced in Section~\ref{subsec:netlist}, a circuit netlist can be naturally modeled as a weighted hypergraph
\begin{equation}
    G_{H} = (V_{H}, E_{H})
    \label{equ:hypergraph}
\end{equation}
where \(V_{H}\) is the set of vertices and \(E_{H}\) is the set of hyperedges, each of which may connect more than two vertices. 
In FPGA context, \(V_{H}\) corresponds to the set of \texttt{SiteInst}s placed on the \texttt{device}, while \(E_{H}\) corresponds to the set of \texttt{Net}s connecting them.
Each hyperedge is incident to one voltage source and one or many voltage sinks, defining the directionality of each hyperedge.
We can then reduce this hypergraph into a graph using one of various net models - star net model, clique net model, Bound2Bound net model, among others.

The star model transforms the hyperedge into a set of 2-pin edges connecting the incident sink pins to the source pin.
The clique net model transforms the hyperdedge into a local fully connected network of 2-pin edges between the vertices incident to the original hyperedge.
This can lead to higher quality results but requires a quadratic growth in edges.
We could preserve the directionality of the edges for possible future optimizations, but since HPWL minimization is not directly affected by the directionality of nets, we can just reduce the directed hypergraph into an undirected graph.
Figure \ref{fig:net_model} shows an example of a hyperedge reduction using the star model and clique model.
The weight of each edge corresponds inversely to the wirelength of the net it represents.

\vspace{1.0cm}
{
    \centering
    \includegraphics[width=\columnwidth]{figures/future_work/net_model.png}
    \captionof{figure}{Hypergraph reduction via Star and Clique net models.}
    \label{fig:net_model}
}


After the hypergraph reduction, most AP approaches then minimize a HPWL cost function using either the sum of linear distances \ref{equ:Manhattan} or the sum of squared distances \ref{equ:Euclidian}.

\begin{equation}
    \boldsymbol{\Phi} (\boldsymbol{x}, \boldsymbol{y}) = \sum_{i,j} w_{i,j} \left( |x_i - x_j| + |y_i - y_j| \right)
    \label{equ:Manhattan}
\end{equation}

\begin{equation}
    \boldsymbol{\Phi} (\boldsymbol{x}, \boldsymbol{y}) = \sum_{i,j} w_{i,j} \left[ (x_i - x_j)^2 + (y_i - y_j)^2 \right]
    \label{equ:Euclidian}
\end{equation}

Here, module placements in the \(x\) and \(y\) directions are captured by the placement vectors \( \boldsymbol{x} = (x_1, x_2, ..., x_n) \) and \( \boldsymbol{y} = (y_1, y_2, ..., y_n) \).
The weight can be calculated depending on the net model used.
Placers using the star net model typically assign the weight as the inverse of the HPWL of the net, as shown in (\ref{equ:weight}).
Placers using the clique model or the Bound2Bound (B2B) model include a \(p-1\) term in the denominator as shown in (\ref{equ:weight_linearized}), where \(p\) is the number of pins on the original hyperedge, to de-emphasize high-fanout nets like the clock and reset nets.


\begin{equation}
    w_{i,j} = \frac{1}{|x_i - x_j|}
    \label{equ:weight}
\end{equation}

\begin{equation}
    w_{i,j} = \frac{1}{(p-1) |x_i - x_j|}
    \label{equ:weight_linearized}
\end{equation}

\begin{equation}
    \boldsymbol{\Phi} (\boldsymbol{x}) = \frac{1}{2} \boldsymbol{x}^T \boldsymbol{Q_x} \boldsymbol{x} + \boldsymbol{c_x}^T \boldsymbol{x} + const.
    \label{equ:quadratic}
\end{equation}



\subsection{Variations on Packing}
Our current placer follows a \textbf{Site-centric} approach that resembles that of the old Xilinx ISE - the predecessor to Vivado used in the 2000s.
These days, Vivado performs \textbf{BEL-centric} placement without necessarily locking \texttt{Cell}s into \texttt{Site}s, allowing for a higher granularity of movement of \texttt{Cell}s. 
Enabling BEL-centric moveme
nt as opposed to Site-centric movement can improve HPWL minimization, but will add much more complexity to the packing and placement process to ensure robustness with respect to hardware constraints, particularly with intra-\texttt{Site} routing.

There is no constraint that forces us to pack \texttt{Cells} into \texttt{SiteInst}s before placement or to follow a strict prepacking-packing-placement flow. 
After implementing a force-directed or analytical placer, we can begin to explore different packing and placement ordering like those studied in Wuxi et al. (2019) \cite{ExplicitPacking}.

{
    \centering
    \includegraphics[width=\columnwidth]{figures/future_work/legalization.png}
    \captionof{figure}{Representative FPGA placement and packing flows. Figure taken from Wuxi et al. (2019), page 1 \cite{ExplicitPacking}}
}
\vspace{0.25cm}

\subsection{Add Hard Macro Support}
fdsa

\subsection{Parallelization}
fdsa




