\section{Placement Results}
\label{sec:results}


We will need an HDL design that utilizes a good mixture of LUTs, FFs, BRAMs, and DSPs at scale to demonstrate the robustness and performance of our placers. 
Any DSP subsystem can serve as a good candidate for such a demonstration. 
Here, we will perform placement on a 2048th-order FIR Filter which was conveniently created as coursework for a VLSI course. 
After passing synthesis, the FIR Filter design calls for the primitive cells shown in the second column of table \ref{fig:utilization}.
We will perform placement of this design on a \texttt{xc7z020} FPGA, which is a mid-ranged Xilinx device containing the available BELs shown in the first column of table \ref{fig:utilization}.
Note that \texttt{LUT5-LUT6} pairs are counted as a single \texttt{LUT}.
% Also note that the number of \texttt{CARRY4} BELs is equal to the total number of \texttt{SLICEL} and \texttt{SLICEM} Sites combined and that the number of \texttt{LUT}s is equal to four times that amount while the number of \texttt{FF}s is equal to eight times that amount.

\vspace{0.5cm}
{
    \centering
    \includegraphics[width=0.8\columnwidth]{figures/results/utilization.png}
    \captionof{figure}{Utilization of BELs on a \texttt{xc7z020} FPGA for a FIR filter design}
    \label{fig:utilization}
}
\vspace{0.5cm}

Now, we present test results for a basic SA placer as well as test results for four simple variations as listed below. 
Note that midpoint simply means centroid.

\begin{itemize}
    \item \texttt{PlacerGreedyRandom}: all undirected random moves with greedy acceptance (always zero temperature)
    \item \texttt{PlacerGreedyMidpoint}: all directed centroid moves with greedy acceptance (always zero temperature)
    \item \texttt{PlacerAnnealRandom}: all undirected random moves with annealing acceptance \textbf{(the basic SA algorithm)}
    \item \texttt{PlacerAnnealMidpoint}: all directed centroid moves with annealing acceptance
    \item \texttt{PlacerAnnealHybrid}: \textbf{Initial:} 50\% random - 50\% centroid moves with annealing acceptance. \textbf{At Near-Zero Temperature:} 100\% centroid moves with annealing acceptance.
\end{itemize}

((Talk about the initial cost.))

Shown in figure \ref{fig:placers_overlay} are the total HPWL curves over number of passes for each of the five placers. 
\texttt{PlacerGreedyMidpoint} (in red) appears to perform the worst by a considerable margin with a final HPWL cost of 590K.
This is likely due to the lack of randomness or lack of diversity in move proposals combined with the persistent greediness of the movement acceptance that prevents the system from escaping local minima.
This conclusion is supported by its HPWL curve (in red) showing a sharp drop in cost that quickly asymptotes just below the 600K mark, suggesting that the system crystallized very quickly into a local minimum and was unable to climb out of it.

Next is \texttt{PlacerAnnealMidpoint} with a final cost of 466K.
Like the previous placer, \texttt{PlacerAnnealMidpoint} lacks the diversity of random moves which contributes to faster crystallization into local minima.
However, this placer likely performed better than the previous as there is still some randomness in the move acceptance evaluation which gives way to occasional hill-climbing. 
Keep in mind that centroid moves do not necessarily decrease system cost. 
The centroid move can decrease the HPWL cost of the nets on connected to the current \texttt{SiteInst}, but can potentially increase the cost of another group of nets in the event of a swap, which can potentially result in an increase in total system cost.

The next three placers performed significantly better than the previous two placers.
Our vanilla SA placer, \texttt{PlacerAnnealRandom} is represented by the green curve.
We can observe that the cost decreases more steadily than the previous two placers, with noticeable spikes in total system cost due to hill-climbing, until settling at the 346K mark as the global temperature cools to zero.

Next is \texttt{PlacerGreedyRandom} which is represented by the purple curve.
We can observe a much steeper initial drop in system cost than the vanilla SA placer due to greedy movement acceptance.
Interestingly, this placer also settled at a lower final system cost of 333K, which is a slight 4\% improvement over vanilla SA.
We originally expected the greediness to be a detriment to finding the global minimum, but due to the inherent randomness of the algorithm, these expectations are never certain.
Broader trends will require many more test trials to observe reliably.

{
    \centering
    \includegraphics[width=\columnwidth]{figures/results/combined_cost_history_linear.png}
    \captionof{figure}{All placers overlaid}
    \label{fig:placers_overlay}
}

The best performing placer was \texttt{PlacerAnnealHybrid}, represented by the blue curve.
We can observe that \texttt{AnnealHybrid} settles at a similar rate with the vanilla \texttt{AnnealRandom}, but eventually settles at a lower minimum at 322K, outperforming even \texttt{GreedyRandom}. 

{
    \centering
    \includegraphics[width=0.7\columnwidth]{figures/placement/cooling_schedule_10000_98.png}
    \captionof{figure}{Cooling Schedule: \(T_0=10000\), \(\alpha=0.98\)}
    \label{fig:cooling_schedule_10000_98}
}

\end{multicols}
{
    \centering
    \includegraphics[valign=t, scale=0.13]{figures/results/PlacerGreedyRandom/random_placement.png}
    \includegraphics[valign=t, scale=0.13]{figures/results/PlacerGreedyRandom/00000010.png}
    \includegraphics[valign=t, scale=0.13]{figures/results/PlacerGreedyRandom/00000100.png}
    \includegraphics[valign=t, scale=0.13]{figures/results/PlacerGreedyRandom/00000299.png}
    \captionof{figure}{\texttt{PlacerGreedyRandom}}
    \label{fig:PGRSnapshots}
}

{
    \centering
    \includegraphics[valign=t, scale=0.13]{figures/results/PlacerGreedyMidpoint/random_placement.png}
    \includegraphics[valign=t, scale=0.13]{figures/results/PlacerGreedyMidpoint/00000010.png}
    \includegraphics[valign=t, scale=0.13]{figures/results/PlacerGreedyMidpoint/00000100.png}
    \includegraphics[valign=t, scale=0.13]{figures/results/PlacerGreedyMidpoint/00000299.png}
    \captionof{figure}{\texttt{PlacerGreedyMidpoint}}
    \label{fig:PGMSnapshots}
}

\newcolumn
{
    \centering
    \includegraphics[valign=t, scale=0.13]{figures/results/PlacerAnnealRandom/random_placement.png}
    \includegraphics[valign=t, scale=0.13]{figures/results/PlacerAnnealRandom/00000010.png}
    \includegraphics[valign=t, scale=0.13]{figures/results/PlacerAnnealRandom/00000100.png}
    \includegraphics[valign=t, scale=0.13]{figures/results/PlacerAnnealRandom/00000299.png}
    \captionof{figure}{\texttt{PlacerAnnealRandom}}
    \label{fig:PARSnapshots}
}

{
    \centering
    \includegraphics[valign=t, scale=0.13]{figures/results/PlacerAnnealMidpoint/random_placement.png}
    \includegraphics[valign=t, scale=0.13]{figures/results/PlacerAnnealMidpoint/00000010.png}
    \includegraphics[valign=t, scale=0.13]{figures/results/PlacerAnnealMidpoint/00000100.png}
    \includegraphics[valign=t, scale=0.13]{figures/results/PlacerAnnealMidpoint/00000299.png}
    \captionof{figure}{\texttt{PlacerAnnealMidpoint}}
    \label{fig:PAMSnapshots}
}

{
    \centering
    \includegraphics[valign=t, scale=0.13]{figures/results/PlacerAnnealHybrid/random_placement.png}
    \includegraphics[valign=t, scale=0.13]{figures/results/PlacerAnnealHybrid/00000010.png}
    \includegraphics[valign=t, scale=0.13]{figures/results/PlacerAnnealHybrid/00000100.png}
    \includegraphics[valign=t, scale=0.13]{figures/results/PlacerAnnealHybrid/00000299.png}
    \captionof{figure}{\texttt{PlacerAnnealHybrid}}
    \label{fig:PAHSnapshots}
}

\begin{multicols}{2}


